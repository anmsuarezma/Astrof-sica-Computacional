\documentclass[11pt]{article}
\pagestyle{plain}
\usepackage{latexsym,exscale,amsfonts,amsmath,amssymb,array}
\usepackage{color}
\usepackage[colorlinks]{hyperref}
\setlength{\topmargin}{-2.3cm}
\setlength{\textheight}{23.8cm}
\setlength{\oddsidemargin}{-0.5cm}
\setlength{\textwidth}{17cm}
\setlength{\parindent}{0cm}
\setlength{\parskip}{.4cm}
\newcommand{\totaldiffx}{\frac{d}{dx}}
\newcommand{\pardiffx}{\frac{\partial}{\partial x}}
\newcommand{\luft}{\:\!}

\usepackage{graphicx}
\usepackage[latin1]{inputenc}
\usepackage{mathpazo}
\usepackage[T1]{fontenc}
\usepackage[comma,numbers,sort&compress]{natbib}


\begin{document}
\begin{center}
\large \bf Computational Astrophysics \rm \\
2019\\
{\small Exercises 14. Burguer's Equation}
\end{center}

\noindent{\bf Burguer's Equation}\\
Burger's equation:
\begin{equation}
\frac{\partial u}{\partial t} + u\frac{\partial u}{\partial x} = 0\,\,,
\end{equation}
 is often used as a first step into hydrodynamics. It is almost
identical to the advection equation treated before, but this time the wave speed is 
NOT a constant $v$ but is
given by the field $u$ itself. This fact may lead to shocks, which are typical in
hydrodynamic situations.

\begin{enumerate}
\item Use Burger's equation to evolve a sine profile,
\begin{equation}
\Psi_0 = \Psi(x,t=0) = \frac{1}{8} \sin \left( \frac{2\pi x}{L} \right)
\end{equation}
in a $[0,L]$ domain with $L=100$. \\
Use ``outflow'' boundary conditions  (these copy the data of the last interior grid point into the
boundary points).
\begin{center}
\begin{minipage}{0.4\textwidth}\includegraphics[width=1\textwidth]{burgers_plot.pdf}
\end{minipage}
\end{center}
\renewcommand{\labelenumi}{(\arabic{enumi})}

Implement the upwind scheme and demonstrate by experiment
that the solution to Burger's equation forms a shock after $t \gtrsim 140$
time has passed. Remember that upwind means ``in the direction opposite of the
velocity''.
  \vspace{0.1cm}
%\item[(1; 20pts)] Using the same code and settings evolve
%\begin{equation}
%\Phi_0(x) = 1-x^2
%\end{equation}
%in the domain [-1:1].
%{\color{red} For this one can compute the solution analytically using the
%characteristics. One has $u(x,t) = u(\eta,0)$ where $\eta$ is found from the
%equation $0 = x - \eta - t u(\eta,0)$ which for the choice of $\Phi_0$ above
%is solvable with solutions of the form $\eta(x,t) = \frac{A}{2t} \pm
%\sqrt{\frac{A^2}{4t^2} - \frac{B(x-t)}{t}}$.}
%  \vspace{0.1cm}

\item Use Burger's equation to evolve a step profile,
\begin{equation}
\Phi_0 = \Phi(x,t=0) = 
\begin{cases}
1 \text{ if } x<0.5 \\
2 \text{ if } x>0.5
\end{cases}
\end{equation}
in the domain $[0,1]$. \\

Implement the upwind scheme and demonstrate by experiment
that the solution to Burger's equation presents rarefaction.

\end{enumerate}

Happy Coding :) !


\end{document}
