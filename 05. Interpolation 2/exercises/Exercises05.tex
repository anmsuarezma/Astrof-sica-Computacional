\documentclass[10pt,letterpaper,notitlepage]{report}
\usepackage[utf8]{inputenc}
\usepackage{amsmath}
\usepackage{amsfonts}
\usepackage{amssymb}
\usepackage{graphicx}
\begin{document}
\title{Computational Astrophysics}
\author{Exercises 5 }
\maketitle

\begin{enumerate}

\item Consider the Lagrangian interpolation of the function  $f(x) = \frac{1}{25 x^2 + 1}$ for $n=6$, $n=8$, $n=10$ and $n=12$ 
done in Exercises 4. Now, discretize the domain with $m=100$ equally spaced points in the interval
  $[-1,1]$. Compute the Error-Norm-2 (EN2), defined as
  \begin{equation*}
    \text{EN2} = \frac{1}{m} \sqrt{ \sum_{i=1}^m 
      \left(\frac{p(x)-f(x)}{f(x)}\right)  ^2 }\,,
  \end{equation*}
 for the 4 cases $n=\{6,8,10,12\}$.

\item Now discretize the same function with $m_2 = 50$ equally spaced points in the interval
  $[-1,1]$. Implement a routine that interpolates $f(x)$ piecewise
  linearly between these $m_2$ data points  and evaluate EN2
  at the $m= 100$ points used above. Compare your result to the
  results of both exercises.
  
  \item Consider once more the function  $f(x) = \frac{1}{25 x^2 + 1}$. Discretize the domain with $m=21$ equally spaced points in the interval $[-1,1]$ and evaluate numerically its first derivative (centered finite difference inside the interval and one-side derivative on the boundaries). \\
Using this information, implement a routine that generates a piecewise cubic Hermite interpolating
polynomial in the interval. \\
Plot the function and the interpolating polynomial.  
\end{enumerate}

Happy Coding !!

\end{document}