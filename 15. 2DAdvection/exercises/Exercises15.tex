\documentclass[11pt]{article}
\pagestyle{plain}
\usepackage{latexsym,exscale,amsfonts,amsmath,amssymb,array}
\usepackage{color}
\usepackage[colorlinks]{hyperref}
\setlength{\topmargin}{-2.3cm}
\setlength{\textheight}{23.8cm}
\setlength{\oddsidemargin}{-0.5cm}
\setlength{\textwidth}{17cm}
\setlength{\parindent}{0cm}
\setlength{\parskip}{.4cm}
\newcommand{\totaldiffx}{\frac{d}{dx}}
\newcommand{\pardiffx}{\frac{\partial}{\partial x}}
\newcommand{\luft}{\:\!}

\usepackage{graphicx}
\usepackage[latin1]{inputenc}
\usepackage{mathpazo}
\usepackage[T1]{fontenc}
\usepackage[comma,numbers,sort&compress]{natbib}


\begin{document}
\begin{center}
\large \bf Computational Astrophysics \rm \\
2019\\
{\small Exercises 15. 2 Dimensional Advection}
\end{center}

\noindent{\bf 2 Dimensional Advection Equation}\\
The two-dimensional linear advection equation is
\begin{equation}
\partial_t a + u \partial_x a + v \partial_y a = 0
\label{eq:advect2d}
\end{equation}
where $u$ is the velocity in the $x$-direction and $v$ is the velocity in
the $y$-direction. 

As a first example in solving  a 2 dimensional partial differential equation, you will advect a Gaussian profile
\begin{equation}
 \Psi_0 = \Psi(x,y,t=0) = e^{-\frac{(x-x_0)^2 + (y-y_0)^2 }{(2 \sigma^2)}},
 \end{equation}
with $x_0 = y_0= 30$, $\sigma = \sqrt{15}$, with positive
velocities $ u = v = 0.8$ in a $[0,100] \times [0,100]$ domain.\\ 
In order to handle the boundaries, we will choose ``outflow'' boundary
conditions, that simply copy the data of the last interior grid point into the
boundary points.




Happy Coding :) !


\end{document}
