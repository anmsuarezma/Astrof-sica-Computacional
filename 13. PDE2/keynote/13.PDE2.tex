\documentclass[11pt]{beamer}
\usetheme{Rochester}
\usepackage[utf8]{inputenc}
\usepackage{amsmath}
\usepackage{amsfonts}
\usepackage{amssymb}
\usepackage{graphicx}
%\author{}
%\title{}
%\setbeamercovered{transparent} 
%\setbeamertemplate{navigation symbols}{} 
%\logo{} 
%\institute{} 
%\date{} 
%\subject{} 
\begin{document}

\begin{frame}
\title{Computational Astrophysics}
\author{E. Larrañaga}
\institute{Observatorio Astronómico Nacional\\
Universidad Nacional de Colombia}
\titlepage
\end{frame}

\begin{frame}{Outline}
\tableofcontents
\end{frame}

\section{Elliptic Partial Differential Equations }
\begin{frame}[fragile]{Elliptic Partial Differential Equations}
The simplest elliptic partial differential equation is the linear Poisson equation.\\
For the Newtonian gravitational potential $\Phi$, this is
\begin{equation}
\nabla^2 \Phi =  4\pi G \rho\,\,.
\label{eq:pde_poisson1}
\end{equation}
\end{frame}


\begin{frame}[fragile]{Poisson Equation with Spherical Symmetry}
Consider a spherically symmetric mass
distribution. The Laplacian
$\nabla^2$ reduces to
\begin{equation}
\begin{aligned}
\nabla^2 [\cdot] &= \frac{1}{r^2} \frac{\partial}{\partial r}\left(r^2 \frac{\partial}{\partial r} [\cdot] \right)\,\,,\\
&= \frac{2}{r}\frac{\partial}{\partial r}[\cdot] + \frac{\partial^2}{\partial r^2} [\cdot]\,\,. 
\end{aligned}
\end{equation}
\end{frame}
  
\section{ODE Method}
\begin{frame}[fragile]{ODE Method}
Since the PDE reduces to just one variable, $r$, we write it as a second-order ODE
\begin{equation}
\frac{d^2 \Phi}{d r^2}  + \frac{2}{r}\frac{d \Phi}{d r}  = 4 \pi G\rho\,\,.
\end{equation}
This equation is reduced to the system
\begin{equation}
\begin{aligned}
\frac{d \Phi}{dr} &= z\,\,,\\
\frac{dz}{dr}  + \frac{2}{r} z &= 4 \pi G \rho\,\,.
\end{aligned}
\end{equation}
\end{frame}

\begin{frame}[fragile]{ODE Method}
In order to solve inside the matter distribution, we will use
inner and outer boundary conditions,
\begin{equation}
\begin{aligned}
\left . \frac{d \Phi}{dr} \right| _{r=0} & = 0\,\,,\\
\Phi(R_\mathrm{surface}) &= - \frac{GM(R_\mathrm{surface})}{R_\mathrm{surface}}\,\,.
\end{aligned}
\end{equation}
\end{frame}

\begin{frame}[fragile]{ODE Method}
In practice, one sets $\Phi(r=0) = 0$ and integrates out. Then, the 
result is corrected by adding a constant term to obtain the outer
boundary condition, which is the analytic value for the
gravitational potential outside a spherical mass distribution.  \\
\bigskip

Remember that this added constant doesn't affect the field value 
because of gauge invariance!
\end{frame}

\section{Matrix Method}
\begin{frame}[fragile]{Matrix Method}
In the matrix method, we discretize Poisson Equation
with centered differences. Then, for an interior (non-boundary) grid point $i$ we have
\begin{equation}
\begin{aligned}
\frac{\partial \Phi }{\partial r} \bigg|_i &\approx \frac{1}{2\Delta r} \left(\Phi_{i+1} - \Phi_{i-1}\right)\,\,,\\
\frac{\partial^2 \Phi }{\partial r^2}  \bigg|_i &\approx \frac{1}{(\Delta r)^2} \left(\Phi_{i+1} - 2 \Phi_i + \Phi_{i-1} \right)\,\,.
\end{aligned}
\end{equation}
Then, Poisson equation is
\begin{equation}
\begin{aligned}
\frac{1}{r_i} \frac{\left(\Phi_{i+1} - \Phi_{i-1}\right)}{\Delta r}  + \frac{\left(\Phi_{i+1} - 2 \Phi_i + \Phi_{i-1} \right)}{(\Delta r)^2}  &= 4\pi G \rho_i\,\,.
\end{aligned}
\end{equation}
\end{frame}

\begin{frame}[fragile]{Matrix Method}
\textbf{Inner boundary condition:}\\
Because of the $1/r$ term in the Laplacian, it is clear that the behavior 
 at $r=0$ must be regularized. \\
 \bigskip
 
 It is possible to use an expansion or we can
stagger the grid so that there is no point exactly at $r=0$, for example 
by moving the entire grid over by $0.5 \Delta r$.  \\
\end{frame}

\begin{frame}[fragile]{Matrix Method}
\textbf{Inner boundary condition:}\\
At the inner boundary, we have the condition $\frac{\partial \Phi}{ \partial r} = 0$, so we assume
\begin{equation}
\Phi_{-1} = \Phi_{0}\,\,.
\end{equation}
\end{frame}

\begin{frame}[fragile]{Matrix Method}
Hence, Poisson equation can be written as linear system
\begin{equation}
J \mathbf{\Phi} = \mathbf{b}\,\,,
\label{eq:pde_poisson2}
\end{equation}
where $\Phi$ = $(\Phi_0, \cdots , \Phi_{n-1})^T$ (for a grid with $n$
points labeled $0$ to $n-1$) and  $\mathbf{b} = 4\pi G
(\rho_0, \cdots , \rho_{n-1})^T$. 
\end{frame}

\begin{frame}[fragile]{Matrix Method}
The matrix $J$ has tri-diagonal form and can be explicitely given as:
\begin{itemize}
\item[(a)] $i=j=0$:
\begin{equation}
J_{00} = - \frac{1}{(\Delta r)^2} - \frac{1}{r_0 \Delta r}\,\,,
\end{equation}
\item[(b)] $i=j$:
\begin{equation}
J_{ij} = \frac{-2}{(\Delta r)^2}\,\,,
\end{equation}
\item[(c)] $i+1=j$:
\begin{equation}
J_{ij} = \frac{1}{(\Delta r)^2} + \frac{1}{r_i \Delta r}\,\,,
\end{equation}
\item[(d)] $i-1=j$:
\begin{equation}
J_{ij} = \frac{1}{(\Delta r)^2} - \frac{1}{r_i \Delta r}\,\,.
\end{equation}
\end{itemize}
\end{frame}

\begin{frame}[fragile]{Matrix Method}
After finding the solution to the linear system, it is neccesary to correct for the analytic outer boundary
value 
\begin{equation}
\Phi(R_\text{surface}) = - \frac{G M}{R_\text{surface}}.
\end{equation}
\end{frame}


\begin{frame}[fragile]{Next Class}
Poisson Equation
\end{frame}

\end{document}
