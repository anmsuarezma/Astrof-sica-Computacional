\documentclass[11pt]{beamer}
\usetheme{Warsaw}
\usepackage[utf8]{inputenc}
\usepackage{amsmath}
\usepackage{amsfonts}
\usepackage{amssymb}
\usepackage{graphicx}
%\author{}
%\title{}
%\setbeamercovered{transparent} 
%\setbeamertemplate{navigation symbols}{} 
%\logo{} 
%\institute{} 
%\date{} 
%\subject{} 
\begin{document}

\begin{frame}
\title{Computational Astrophysics}
\author{E. Larrañaga}
\institute{Observatorio Astronómico Nacional\\
Universidad Nacional de Colombia}
\titlepage
\end{frame}

\begin{frame}{Outline}
\tableofcontents
\end{frame}

\section{Fundamentals of Python}

\subsection{Print and Importing Modules}
\begin{frame}[fragile]{Hello World!}
\begin{semiverbatim}
# Print the Hello World! statment
print("Hello World!")
\end{semiverbatim}
\end{frame}

\begin{frame}[fragile]{Importing a Module}
\begin{semiverbatim}
# import the math module
import math

x = math.pi  
print("An approximte value of pi is \%8.5f" \% x)

\end{semiverbatim}
\end{frame}

\begin{frame}[fragile]{Print Formatting}
\footnotesize
\begin{semiverbatim}
\%[width][.precision]type 
\end{semiverbatim}
\pause
\begin{semiverbatim}
[width] : total number of digits (including the decimal point)
[.precision] : digits in the decimal part of the number
\end{semiverbatim}
\pause
\begin{semiverbatim}
type: 
d (Integer)
f (Float)
e (Scientific Notation)
g (Same as "e" if exponent is greater than -4 or less than precision, 
	"f" otherwise)
	
...
\end{semiverbatim}
\end{frame}

\begin{frame}[fragile]{Importing Modules I}
\begin{semiverbatim}
# import the sys and math modules
import sys , math

x = math.pi / 2.0 
print(math.sin(x))

sys.exit()       # leave program here 

print("This will NOT print!!")

\end{semiverbatim}
\end{frame}


\begin{frame}[fragile]{Importing Modules II}
\begin{semiverbatim}
# import the sys and math modules
import sys 
import math as m

x = m.pi / 2.0 
print(m.sin(x))

sys.exit()       # leave program here 

print("This will not print!!")

\end{semiverbatim}
\end{frame}

\begin{frame}[fragile]{Importing Modules III}
\begin{semiverbatim}
# import the sys and math modules
import sys 
from math import pi, sin

x = pi / 2.0 
print(sin(x))

sys.exit()       # leave program here 

print("This will not print!!")

\end{semiverbatim}
\end{frame}

\begin{frame}[fragile]{Importing Modules IV}
\begin{semiverbatim}
# import the sys and math modules
import sys 
from math import * 

x = pi / 2.0 
print(sin(x))

sys.exit()       # leave program here 

print("This will not print!!")

\end{semiverbatim}
\end{frame}


\subsection{Simple Math}
\begin{frame}[fragile]{Simple Math I}
\small
\begin{semiverbatim}
import math as m

# Simple math operations
a = 2.0
b = 1.0
c = a + b
d = c**2 		# square c
e = m.sqrt(c) 	# take the sqrt of c 
f = d/c 		# divide d by c

# Print results 
print(a,b,c) 
print(d)
print(e)
print(f)
\end{semiverbatim}
\end{frame}

\begin{frame}[fragile]{Simple Math II}
\footnotesize
\begin{semiverbatim}
import math as m
# Trigonometry 
pi = m.pi
theta = 2*pi
print(m.sin(theta)) 
print(m.cos(theta))
print(m.tan(theta))

# Other Operations
a = 5.0
print(m.exp(a))
print(m.log(a))
print(m.sinh(a))
\end{semiverbatim}
\end{frame}


\section{Numerical Python Module Numpy}
\begin{frame}[fragile]{Numpy Arrays I}
\footnotesize
\begin{semiverbatim}
import numpy as np

a = np.array([0.0, 1.0, 2.0, 3.0])
b = np.arange(10)
c = np.zeros(10,float)
d = np.zeros(10,int)
e = np.ones(10)
f = np.random.random(10)

print(a)
print(b)
print(c)
print(d)
print(e)
print(f)
\end{semiverbatim}
\end{frame}

\begin{frame}[fragile]{Numpy Arrays II}
\footnotesize
\begin{semiverbatim}
import numpy as np

a = np.array([[1, 2, 3], [4, 5, 6],[7, 8, 9]])
b = np.array([np.arange(3), np.arange(3),np.arange(3)])
c = np.zeros([3, 3],float)
d = np.ones([3, 3])
e = np.random.random([3, 3])

print(a)
print(b)
print(c)
print(d)
print(e)
\end{semiverbatim}
\end{frame}

\begin{frame}[fragile]{Numpy Arrays Functions}
\footnotesize
\begin{semiverbatim}
import numpy as np

a = np.random.random([3, 4])

print(a)
print(np.ndim(a))    # Number of dimensions
print(len(a))            # Size of the first dimension
print(len(a[0, :]))    # Size of the second dimension
print(a.max())         # Max entry
print(a.min())          # Min entry
print(a.sum())         # Sum of all the elements
print(a[0,:].sum())  # Sum of first row
print(a[:,0].sum())  # Sum of first column
\end{semiverbatim}
\end{frame}


\section{Conditional Statements and Indentation}
\begin{frame}[fragile]{Conditional Statements and Indentation I}
\footnotesize
\begin{semiverbatim}
import numpy as np

# get a random number in the range [0.0,1.0)
a = np.random.random()

if a < 0.5:
    print ("a = \%5.6g is < 0.5!" \% a) 
    print ("Small a!")
else:
    print ("a = \%5.6g is >= 0.5!" \% a) 
    print ("Large a!")
\end{semiverbatim}
\end{frame}

\begin{frame}[fragile]{Conditional Statements and Indentation II}
\footnotesize
\begin{semiverbatim}
import numpy as np

# get a random number in the range [0.0,1.0)
a = np.random.random()

if a < 0.3:
    print("a = \%5.6g is < 0.3 " \% a) 
    print("Small a!")
elif a>= 0.3 and a<0.6:
    print("0.3 <= a = \%5.6g < 0.6 " \% a) 
    print("Medium a!")
else:
    print("a = \%5.6g is >= 0.6 " \% a) 
    print("Large a!")

\end{semiverbatim}
\end{frame}

\subsection{Loops}
\begin{frame}[fragile]{Loops I}
\begin{semiverbatim}
for i in range(100): 
    print(i)

\end{semiverbatim}
\end{frame}

\begin{frame}[fragile]{Loops II}
\begin{semiverbatim}
for i in range(1,100,2): 
    print(i)

\end{semiverbatim}
\end{frame}

\begin{frame}[fragile]{Loops III}
\begin{semiverbatim}
i = 0
while i<100: 
    print(i)
    i = i + 1

\end{semiverbatim}
\end{frame}


\section{Introduction to Functions}
\begin{frame}[fragile]{Functions I}
\begin{semiverbatim}
def myfunction(x):
    return x**2 + 5

a = 2.0
print(myfunction(a))

print(myfunction(3.5))
\end{semiverbatim}
\end{frame}

\begin{frame}[fragile]{Functions II}
\begin{semiverbatim}
def myfunction(x,y):
	return x**2, x + y

a = 2
b = 3

c,d = myfunction(a,b)

print(c)
print(d)
\end{semiverbatim}
\end{frame}

\section{Basic File I/O}
\subsection{Reading Files}
\begin{frame}[fragile]{Reading Files I}
\begin{semiverbatim}
# Open the file
infile = open("example_file1.txt","r")

# Read the data
indata = infile.readlines()

# Close the file again
infile.close()

print(indata)
\end{semiverbatim}
\end{frame}

\begin{frame}[fragile]{Reading Files II}
\begin{semiverbatim}
import numpy as np

# Open the file
infile = open("example_file2.txt","r")

# Read the data
indata = infile.readlines()

# Close the file again
infile.close()

print(indata)
\end{semiverbatim}
\end{frame}

\begin{frame}[fragile]{Reading Files II (cont.)}
\begin{semiverbatim}
# get number of data lines in the file 
# (-1, because first line is a comment) 
n = len(indata) - 1

# allocate two numpy arrays 
time = np.zeros(n)
data = np.zeros(n)
\end{semiverbatim}
\end{frame}

\begin{frame}[fragile]{Reading Files II (cont.)}
\begin{semiverbatim}
# parse the data (separated by a comma in the file)
for i in range(n):
    splitline = indata[i+1][:-1].split(',')
    time[i] = float(splitline[0])
    data[i] = float(splitline[1])

print(time)
print(data)
\end{semiverbatim}
\end{frame}

\subsection{Writting Files}
\begin{frame}[fragile]{Writting Files}
\tiny
\begin{semiverbatim}
import numpy as np

# open the file for writting
outfile = open("outfile.txt","w")

# how many lines do we want to output 
n = 100

# write a header
headerstring = "# This is header for the data\\n" 
outfile.write(headerstring)

# write some random data with a formatting
for i in range(n):
    ran1 = np.random.random()
    ran2 = np.random.random()
    outstring = "\%10.10E,\%10.10E\\n" \% (ran1,ran2) 
    outfile.write(outstring)

#close the file
outfile.close()
\end{semiverbatim}
\end{frame}

\begin{frame}[fragile]{Reading and Writting Files with Numpy}
\small
\begin{semiverbatim}
import numpy as np

# read file
data = np.loadtxt("example_file3.txt",comments="#")
print(data)

# do something with data, e.g., square each entry 
data[: , :] = data [: , :]**2
print(data)

# write result 
np.savetxt("outfile3.txt",data,fmt="\%10.10E",\\
    header="This is a data file")
\end{semiverbatim}
\end{frame}




\end{document}