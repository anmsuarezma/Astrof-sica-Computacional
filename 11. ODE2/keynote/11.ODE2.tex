\documentclass[11pt]{beamer}
\usetheme{Rochester}
\usepackage[utf8]{inputenc}
\usepackage{amsmath}
\usepackage{amsfonts}
\usepackage{amssymb}
\usepackage{graphicx}
%\author{}
%\title{}
%\setbeamercovered{transparent} 
%\setbeamertemplate{navigation symbols}{} 
%\logo{} 
%\institute{} 
%\date{} 
%\subject{} 
\begin{document}

\begin{frame}
\title{Computational Astrophysics}
\author{E. Larrañaga}
\institute{Observatorio Astronómico Nacional\\
Universidad Nacional de Colombia}
\titlepage
\end{frame}

\begin{frame}{Outline}
\tableofcontents
\end{frame}

\section{Ordinary Differential Equations with Boundary Conditions}
\begin{frame}[fragile]{Ordinary Differential Equations with Boundary Conditions}
A boundary
value problem consists of finding a solution of an ODE
in an interval $[a,b]$ that satisfies constraints at both
ends (boundary conditions).

\textbf{Example}
\begin{equation}
y'' = f(x,y,y')\,\,, 
\hspace{0.5em} y(a) = A\,\,,
\hspace{0.5em} y(b) = B\,\,, 
\hspace{0.5em} \text{and } x \in [a,b]\,\,.
\label{eq:ode_bvp1}
\end{equation}
\end{frame}


\section{Shooting Method}
\begin{frame}[fragile]{Shooting Method}
The shooting method solves a BVP  by transforming it into an initial value problem by
making an educated guess on unknown inner boundary conditions. Then, we iterate
 until a modified guessed inner boundary condition leads to
the correct known outer boundary value.
\end{frame}

\begin{frame}[fragile]{Shooting Method}
\textbf{Example}\\
 
Given the system in Eq.~\ref{eq:ode_bvp1}, the value of $y'(a)$ is
unknown.\\

We can make a guess $y'(a) = z$ to write $z' = f(a,y(a),y'(a))$. \\

This reduces
the second-order problem to two first-order problems. 
\end{frame}

\begin{frame}[fragile]{Shooting Method}
Then we just need to integrate
the two ODEs out to $b$. Since we have chosen $z$, we have now solved
$y=y(x,z)$, but our goal is to find $y$ such that $y(b,z) = B$.

In other words, we can define a new function 
\begin{equation}
\Phi(z) = y(b,z) - B
\end{equation}
and search for a $z$ so that $\Phi(z) = 0$. Hence, we are looking
for the root of $\Phi(z)$!
\end{frame}

\begin{frame}[fragile]{Shooting Method}
The full shooting algorithm for $y'' = f(x,y,y')$ goes as

\begin{enumerate}
\item Guess a starting value $z_0 = y'(a)$, set the iteration counter 
$i = 0$.
\item Compute $y = y(x,z_i)$ by integrating the IVP.
\item Compute $\Phi(z_i) = y(b,z_i) - B$. If $z_i$ does not
      give a sufficiently accurate solution of the full problem,
      increment $i$ to $i+1$ and find  a value for
      $z_{i+1}$ using a root finder on $\Phi(z_i) = 0$. 
      Then go back to (2).
\end{enumerate}
\end{frame}

\begin{frame}[fragile]{Shooting Method}
Note that one typically ends up with the secant method, since the
derivative of $\Phi(z)$ is not known in the general case and one is
stuck with having to numerically compute it. For this, at least two
guesses for $z$ are needed.
\end{frame}

\subsection{Finite-Difference Method}
\begin{frame}[fragile]{Finite-Difference Method}
BVPs of the kind given by Eq.~(\ref{eq:ode_bvp1}) can be solved
by Taylor expanding the ODE itself to linear
order (assuming there are no non-linearities in y and y'):
\begin{equation}
y'' = g(x) - p(x)y' - q(x) y\,\,,
\label{eq:ode_bvp2}
\end{equation}
where $g(x)$, $p(x)$, and $q(x)$ are functions of $x$ only and the sign
convention is arbitrary.
\end{frame}

\begin{frame}[fragile]{Finite-Difference Method}
We can now discretize $y'$ and $y''$ on an evenly spaced grid with
step size $h$,
\begin{equation}
\begin{aligned}
y'(x_i) &= \frac{y(x_{i+1}) - y(x_{i-1})}{2h}\,\,,\\
y''(x_i) &= \frac{y_{i+1} + y_{i-1} - 2 y_i}{h^2}\,\,,
\end{aligned}
\end{equation}
where $x_i = a + i h$, $(i = 0, ... , n+1)$, and 
\begin{equation}
h = \frac{b-a}{n+1}.
\end{equation}
\end{frame}

\begin{frame}[fragile]{Finite-Difference Method}
The system is a tri-diagonal matrix of dimension $n \times n$:
\tiny
\begin{equation}
\begin{pmatrix}
-2+h^2q_1 & 1+ \frac{h}{2}p_1& 0 &\multicolumn{3}{c}{\cdots} &0\\
1-\frac{h}{2} p_2&\ddots&\ddots & 0 &\multicolumn{2}{c}{\cdots} & 0\\
0 & \ddots & \ddots & \ddots & 0 & \cdots & \vdots\\
\vdots & 0 & \ddots & \ddots & \ddots & 0 & \vdots \\
\vdots & \vdots & 0 & \ddots & \ddots & \ddots & 0\\
\vdots & \vdots & \vdots& 0& \ddots & \ddots & 1+\frac{h}{2}p_{n-1}\\
\vdots & 0 & 0 & \cdots & 0 & 1-\frac{h}{2} p_n & -2+h^2 q_n\\
\end{pmatrix}
\begin{pmatrix}
y_1\\\vdots \\\vdots \\\vdots \\\vdots \\\vdots \\ y_n
\end{pmatrix}
= \begin{pmatrix}
h^2 g_1 - A(1-\frac{h}{2}p_1)\\
h^2 g_2\\
\vdots\\
\vdots\\
\vdots\\
h^2 g_{n-1}\\
h^2 g_n - B(1+\frac{h}{2}p_n)\\
\end{pmatrix}
\end{equation}
\end{frame}



\begin{frame}[fragile]{Next Class}
Partial Differential Equations.
\end{frame}

\end{document}
