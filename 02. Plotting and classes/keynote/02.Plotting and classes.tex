\documentclass[11pt]{beamer}
\usetheme{Warsaw}
\usepackage[utf8]{inputenc}
\usepackage{amsmath}
\usepackage{amsfonts}
\usepackage{amssymb}
\usepackage{graphicx}
%\author{}
%\title{}
%\setbeamercovered{transparent} 
%\setbeamertemplate{navigation symbols}{} 
%\logo{} 
%\institute{} 
%\date{} 
%\subject{} 
\begin{document}

\begin{frame}
\title{Computational Astrophysics}
\author{E. Larrañaga}
\institute{Observatorio Astronómico Nacional\\
Universidad Nacional de Colombia}
\titlepage
\end{frame}

\begin{frame}{Outline}
\tableofcontents
\end{frame}

\section{Basics of Plotting with matplotlib}

\subsection{Data Plotting}
\begin{frame}[fragile]{Data Plotting}
\begin{semiverbatim}
import matplotlib.pyplot as plt 
import numpy as np

file_name = "plotdata.txt" 
data = np.loadtxt(file_name, comments="#") 
x = data[:,0]
y = data[:,1]

# plot the data 
plt.plot(x,y) 
plt.show()
\end{semiverbatim}
\end{frame}

\begin{frame}[fragile]{Data Plotting Options}
\begin{semiverbatim}
import matplotlib.pyplot as plt 
import numpy as np

file_name = "plotdata.txt" 
data = np.loadtxt(file_name, comments="#") 
x = data[:,0]
y1 = data[:,1]
y2 = data[:,2]

# make the plots
fig, ax = plt.subplots()
p1 = ax.plot(x, y1, "r", linewidth = 2)
p2 = ax.plot(x, y2, "b", linewidth = 2)
\end{semiverbatim}
\end{frame}

\begin{frame}[fragile]{Data Plotting Options (cont.)}
\begin{semiverbatim}
# labels of the plot
ax.set(xlabel="X", ylabel="Y",
    title="Title of the plot")

# set x and y ranges 
plt.xlim(min(x),max(x)) 
plt.ylim(min(y1*1.05),max(y1*1.05))

# plot's grid
ax.grid()
\end{semiverbatim}
\end{frame}

\begin{frame}[fragile]{Data Plotting Options (cont.)}
\begin{semiverbatim}
# show the plot
plt.show()

# save the plot as a pdf or png
fig.savefig("simpleplot.pdf")
fig.savefig("simpleplot.png")
\end{semiverbatim}
\end{frame}

\subsection{Curve and Function Plotting}
\begin{frame}[fragile]{Curve Plotting I}
\tiny
\begin{semiverbatim}
import numpy as np
import matplotlib.pyplot as plt

def f(t):
    return t**2*np.exp(-t**2)

# range of the independent variable
# 50 points between 0 and 3
t = np.linspace(0, 3, 50)

# values of the function for the numbers in the range of t
y = np.zeros(len(t))
for i in range(len(t)):
    y[i] = f(t[i])

# plot and label
fig, ax = plt.subplots()
ax.plot(t, y, label='t^2*exp(-t^2)')
ax.legend()

# show the plot
plt.show()
\end{semiverbatim}
\end{frame}

\begin{frame}[fragile]{Curve Plotting II}
\tiny
\begin{semiverbatim}
import numpy as np
import matplotlib.pyplot as plt

def f(t):
    return t**2*np.exp(-t**2)

def g(t):
    return t*np.sin(2*t)

# range of the independent variable
# 50 points between 0 and 3
t = np.linspace(0, 3, 50)

# values of the function for the numbers in the range of t
y1 = f(t)
y2 = g(t)

# plot and label of the curve
fig, ax = plt.subplots()
ax.plot(t, y1, 'r-', label='f(t) = t^2*exp(-t^2)')
ax.plot(t, y2, 'bo', label='g(t) = t*sin(2t)')

# labels of the plot
ax.set(xlabel="t", ylabel="f(t), g(t)",
       title="Two curves")
ax.legend()

# show the plot
plt.show()
\end{semiverbatim}
\end{frame}

\begin{frame}[fragile]{Curve Plotting III}
\tiny
\begin{semiverbatim}
import numpy as np
import matplotlib.pyplot as plt

def f(t):
    return t**2*np.exp(-t**2)

def g(t):
    return t*np.sin(2*t)

# range of the independent variable
t = np.linspace(0, 3, 50)
# values of the function for the numbers in the range of t
y1 = f(t)
y2 = g(t)

# separate figures
plt.figure()

# First subplot
plt.subplot(2, 1, 1)
plt.plot(t, y1, 'r-', label='f(t) = t^2*exp(-t^2)')
plt.legend()
# Second subplot
plt.subplot(2, 1, 2)
plt.plot(t, y2, 'bo', label='g(t) = t*sin(2t)')
plt.legend()

# show the plot
plt.show()
\end{semiverbatim}
\end{frame}

\section{Classes}
\begin{frame}[fragile]{Defining a Class}
\begin{enumerate}
\pause
\item Attributes
\item 
\end{enumerate}
\pause

\tiny
\begin{semiverbatim}
class Planet(object):
    def __init__ (self, planet_name, orbit_period, mass):
        self.planet_name = planet_name 
        self.orbit_period = rev_period      # in Earth years
        self.mass = mass                    # in units of Earth mass
\pause
\end{semiverbatim}
Class Initialization
\end{frame}

\begin{frame}[fragile]{Defining a Class}
\begin{enumerate}
\item Attributes
\item 
\end{enumerate}
\tiny
\begin{semiverbatim}
class Planet(object):
    def __init__ (self, planet_name, orbit_period, mass):
        self.planet_name = planet_name 
        self.orbit_period = rev_period      # in Earth years
        self.mass = mass                    # in units of Earth mass
\pause

# defining Mars as a planet 
mars = Planet("Mars", 1.88, 0.107)

# attributes of Mars 
print(mars.rev_period)
print(mars.mass)
\end{semiverbatim}
\end{frame}




\begin{frame}[fragile]{Defining a Class}
\begin{enumerate}
\item Attributes
\item Methods
\end{enumerate}
\pause

\tiny
\begin{semiverbatim}
class Planet(object):
    def __init__ (self, planet_name, orbit_period, mass):
        self.planet_name = planet_name 
        self.orbit_period = orbit_period  # in Earth years
        self.mass = mass # in units of Earth mass
\pause    
    def semimajoraxis(self):
        '''
        returns the semimajor axis of the planet in AU
        calculated using Kepler's third law 
        '''
        return (self.orbit_period**(2./3.))

\end{semiverbatim}
\end{frame}

\begin{frame}[fragile]{Defining a Class}
\begin{enumerate}
\item Attributes
\item Methods
\end{enumerate}

\tiny
\begin{semiverbatim}
# defining Mars as a planet 
mars = Planet("Mars", 1.88, 0.107)

# attributes of Mars 
print(mars.orbit_period)
print(mars.mass)
print(mars.semimajoraxis())

\pause
# defining Earth as a planet 
earth = Planet("Earth", 1., 1.)

# attributes of Earth 
print(earth.orbit_period)
print(earth.mass)
print(earth.semimajoraxis())
\end{semiverbatim}
\end{frame}

\subsection{Subclasses}
\begin{frame}[fragile]{Defining a Subclass}
\tiny
\begin{semiverbatim}
import numpy as np

class Planet(object):
    def __init__ (self, planet_name, orbit_period, mass):
        self.name = planet_name 
        self.orbit_period = orbit_period  # in Earth years
        self.mass = mass # in units of Earth mass
    
    def semimajoraxis(self):
        '''
        returns the semimajor axis of the planet in AU
        calculated using Kepler's third law 
        '''
        return (self.orbit_period**(2./3.))
\pause
class Dwarf(Planet):
    def description(self):
        return self.name + " is a dwarf planet with a mass of " + str(self.mass) + " Earth masses."
\pause
# defining Pluto as a dwarf planet 
pluto = Dwarf("Pluto", 248.00, 0.00218)

# attributes of Pluto 
print(pluto.orbit_period)
print(pluto.mass)
print(pluto.semimajoraxis())
print(pluto.description())
\end{semiverbatim}
\end{frame}

\section{Modules}
\begin{frame}[fragile]{Writing and Using a Module}
mymodule.py

\tiny
\begin{semiverbatim}
def BalmerLines(n):
    '''
    returns the value of the wavelenght lambda (in meters) 
    for a given value of n in the Balmer series
    '''
    if n < 3:
        return None
    else:
        R = 1.09677583E7 # in untis of m^-1
        lambda_inv = R * (1/4 - 1/n**2)
        return 1/lambda_inv
\end{semiverbatim}
\end{frame}

\begin{frame}[fragile]{Writing and Using a Module}
usingmymodule.py

\tiny
\begin{semiverbatim}
import mymodule as mym

print("\\nn        lambda (m)")
for n in range(2,16):
    print (n,"        ", mym.BalmerLines(n))

print()
\end{semiverbatim}
\end{frame}

\end{document}