\documentclass[11pt]{beamer}
\usetheme{Rochester}
\usepackage[utf8]{inputenc}
\usepackage{amsmath}
\usepackage{amsfonts}
\usepackage{amssymb}
\usepackage{graphicx}
%\author{}
%\title{}
%\setbeamercovered{transparent} 
%\setbeamertemplate{navigation symbols}{} 
%\logo{} 
%\institute{} 
%\date{} 
%\subject{} 
\begin{document}

\begin{frame}
\title{Computational Astrophysics}
\author{E. Larrañaga}
\institute{Observatorio Astronómico Nacional\\
Universidad Nacional de Colombia}
\titlepage
\end{frame}

\begin{frame}{Outline}
\tableofcontents
\end{frame}

\section{Partial Differential Equations }
\begin{frame}[fragile]{Partial Differential Equations}
\begin{itemize}
\item A PDE is a relation between the partial derivatives of an unknown
function and the independent variables.
\item The order of the highest derivative sets the \emph{order} of the PDE.
If the highest derivative is a second derivative, we are dealing with
a second-order PDE.
\item A PDE is \emph{linear} if it is of the first degree in the dependent 
variable (i.e. the unknown function) and in its partial derivatives.
\item If each term of a PDE contains either the dependent variable
or one of its partial derivatives, the PDE is called \emph{homogeneous}.
Otherwise it is \emph{non-homogeneous}.
\end{itemize}
\end{frame}


\begin{frame}[fragile]{Types of Partial Differential Equations}
There are three general types of PDEs
\begin{itemize}
\item hyperbolic
\item parabolic
\item elliptic
\end{itemize}  

Not all PDEs fall into one of these three types, but many
PDEs used in practice do. \\

These classes of PDEs model different sorts
of phenomena, display different behavior, and require different numerical
techniques for their solution.\\

It is not always straighforward to see (to show and proof) what type
of PDE a given PDE one is dealing with is. 
\end{frame}

\begin{frame}[fragile]{Linear Second Order Differential Equation}
\begin{equation}
a \partial^2_{xx} u + b \partial^2_{xy} u + c \partial^2_{yy} u + d \partial_x u + e \partial_y u + f u = g\,\,,
\end{equation}
This equation is straightforwardly categorized based on the discriminant,
\begin{equation}
b^2 - 4ac \left\{ \begin{array}{lcr}
< 0 & \rightarrow & \text{elliptic},\\
= 0 & \rightarrow & \text{parabolic},\\
> 0 & \rightarrow & \text{hyperbolic}.
\end{array}\right.
\end{equation}
The names come from analogy with conic sections in the theory of
ellipses.
\end{frame}

\section{Hyperbolic PDEs}
\begin{frame}[fragile]{Hyperbolic PDEs}
Hyperbolic equations in physics and astrophysics
describe \emph{dynamical} processes and systems that generally start
at some initial time $t_0=0$ with some initial conditions. \\
The equations are then integrated in time. \\
\bigskip

The prototypical linear second-order hyperbolic equation is the
homogeneous wave equation,
\begin{equation}
c^2 \partial^2_{xx} u - \partial^2_{tt} u = 0\,\,,
\end{equation}
where $c$ is the wave speed.
\end{frame}

\begin{frame}[fragile]{Hyperbolic PDEs}
Another  class of hyperbolic equations are the \emph{first-order
hyperbolic systems}. In one space dimension and assuming a linear
problem, we have
\begin{equation}
\partial_t u + A \partial_x u = 0\,\,,
\label{eq:pde_linear1}
\end{equation}
where $u(x,t)$ is the state vector with $s$ components and $A$
is a $s \times s$ matrix. \\
The problem is called \emph{hyperbolic} if
$A$ has only real eigenvalues and is diagonizable, i.e., has a complet set
of linearly independent eigenvectors so that one can construct a 
matrix
\begin{equation}
\Lambda = Q^{-1} A Q\,\,,
\end{equation}
where $\Lambda$ is diagonal and has real numbers on its diagonal. 
\end{frame}

\begin{frame}[fragile]{Hyperbolic PDEs}
An example
of these equations is the linear \emph{advection equation}:
\begin{equation}
\partial_t u + v \partial_x u = 0\,\,,
\end{equation}
which is first order, linear, and homogeneous.
\end{frame}

\begin{frame}[fragile]{Hyperbolic PDEs}
Other example is given by the non-linear first-order systems. Consider the equation
\begin{equation}
\partial_t u + \partial_x F(u) = 0\,\,,
\end{equation}
where $F(u)$ is the \emph{flux} and may or may not be non-linear in $u$.
We can re-write this PDE in \emph{quasi-linear} form, by introducing
the Jacobian
\begin{equation}
\bar{A} = \frac{\partial F}{\partial u}\,\,,
\end{equation}
and writing
\begin{equation}
\partial_t u + \bar{A}\partial_x u = 0\,\,.
\label{eq:pde_quasilin1}
\end{equation}

This PDE is hyperbolic if $\bar{A}$ has real eigenvalues and is
diagonizable. \\
\bigskip

The \emph{equations of hydrodynamics} are a key example
of a non-linear, first-order hyperbolic PDE system.
\end{frame}

\begin{frame}[fragile]{Hyperbolic PDEs Conditions}
 \textbf{Initial Conditions for Hyperbolic Problems} \\
One must specify $u(x,t=0)$ (at all $x$) and (order 1)
time derivatives.
\vskip.2cm
\noindent {\bf Boundary Conditions for Hyperbolic Problems}

\noindent One must specify either von Neumann, Dirichlet, or Robin boundary
conditions:
\begin{enumerate}
\item[1] {\bf Dirichlet Boundary Conditions}.  \\
\begin{equation}
\begin{aligned}
u(x=0,t) &= \Phi_1(t)\,\,,\\
u(x=L,t) &= \Phi_2(t)\,\,.
\end{aligned}
\end{equation}
\end{enumerate}
\end{frame}

\begin{frame}[fragile]{Hyperbolic PDEs Conditions}
\begin{enumerate}
\item[2] {\bf von Neumann Boundary Conditions}.\\ 
      \begin{equation}
      \begin{aligned}
      \partial_x u(x=0,t) &= \Psi_1(t)\,\,,\\
      \partial_x u(x=L,t) &= \Psi_2(t)\,\,.
      \end{aligned}
      \end{equation}
      Note that in a multi-D problem $\partial_x$ turns into
      the derivative in the direction of the normal to the boundary.

\item[3] {\bf Robin Boundary Conditions}.\\  $a_1, b_1, a_2, b_2$ be
      real numbers, not $a_i = b_i = 0$. \\
      \begin{equation}
      \begin{aligned}
      a_1 u(x=0,t) + b_1 \partial_x u(x=0,t) &= \Psi_1(t)\,\,,\\
      a_2 u(x=L,t) + b_2 \partial_x u(x=L,t) &= \Psi_2(t)\,\,.
      \end{aligned}
      \end{equation}
      Dirichlet and von Neuman boundary conditions are recovered if either
      both $a_i$ or both $b_i$ vanish.
      Note that in a multi-D problem $\partial_x$ turns into
      the derivative in the direction of the normal to the boundary.
\end{enumerate}
\end{frame}


\section{Parabolic PDEs}
\begin{frame}[fragile]{Parabolic PDEs}
Parabolic PDEs describe processes that are slowly changing, such as
the slow diffusion of heat in a medium, sediments in ground water, and
radiation in an opaque medium. Parabolic PDEs are second order and
have generally the shape
\begin{equation}
\partial_t u - k \partial^2_{xx} u = f\,\,.
\label{eq:pde_para1}
\end{equation}
\end{frame}

\begin{frame}[fragile]{Parabolic PDEs Conditions}
\noindent {\bf Initial Conditions for Parabolic Problems}

\noindent One must specify $u(x,t=0)$ at all $x$.


\vskip.2cm
\noindent {\bf Boundary Conditions for Parabolic Problems}

\noindent Dirichlet, von Neumann or Robin boundary conditions.
If the boundary conditions are independent of time, the system will
evolve towards a steady state ($\partial_t u = 0$). In this case, 
one may set $\partial_t u = 0$ for all times and treat Eq.~(\ref{eq:pde_para1})
as an elliptic equation.
\end{frame}


\section{Elliptic PDEs}
\begin{frame}[fragile]{Elliptic PDEs}
Elliptic equations describe systems that are static, in steady state
and/or in equilibrium. There is no time dependence. A typical elliptic
equation is the Poisson equation:
\begin{equation}
\nabla^2 u = f\,\,,
\end{equation}
which one encounters in Newtonian gravity and in electrodynamics.
$\nabla^2$ is the Laplace operator, and $f$ is a given scalar function
of position. Elliptic problems may be linear ($f$ does not depend on
$u$ or its derivatives) or non-linear ($f$ depends on $u$ or its
derivatives).
\end{frame}

\begin{frame}[fragile]{Elliptic PDEs Conditions}
\noindent {\bf Initial Conditions for Elliptic Problems}

\noindent Do not apply, since there is no time dependence.


\vskip.2cm
\noindent {\bf Boundary Conditions for Elliptic Problems}

\noindent Dirichlet, von Neumann or Robin boundary conditions.
\end{frame}

\section{Numerical Methods for PDEs}
\begin{frame}[fragile]{Numerical Methods for PDEs}
There is no such thing as a general robust method for the solution of 
generic PDEs. Each type (and each sub-type) of PDE requires a different
approach. Real-life PDEs may be of mixed type or may have special 
properties that require knowledge about the underlying physics for their
successful solution.\\

\bigskip
There are three general classes of approaches to solving PDEs

\end{frame}

\begin{frame}[fragile]{Numerical Methods for PDEs}
\textbf{1. Finite Difference Methods.}\\
      
      The differential operators are approximated using their finite-difference
      representation on a given grid. A sub-class of finite-difference methods,
      so-called finite-volume methods, can be used for PDEs arising from
      conservation laws (e.g., the hydrodynamics equations).

      Finite difference/volume methods have polynomial convergence for
      smooth functions.
\end{frame}

\begin{frame}[fragile]{Numerical Methods for PDEs}
 \textbf{2. Finite Element Methods.}\\
      The domain is divided into cells (``\emph{elements}''). The solution
      is represented as a single function (e.g., a polynomial) on each
      cell and the PDE is transformed to an algebraic problem for the matching
      conditions of the simple functions at cell interfaces.

      Finite element methods can have polynomial or exponential convergence
      for smooth functions.
\end{frame}

\begin{frame}[fragile]{Numerical Methods for PDEs}
 \textbf{3. Spectral Methods.}\\
      The solution is represented by a linear combination of known
      functions (e.g. trigonometric functions or special
      polynomials). The PDE is transformed to a set of algebraic
      equations (or ODEs) for the amplitudes of the component
      functions. A sub-class of these methods are the collocation
      methods. In them, the solution is represented on a grid and the
      spectral decomposition of the solution in known functions is
      used to estimate to a high degree of accuracy the partial
      derivatives of the solution on the grid points.
\end{frame}

\section{Linear Advection Equation}
\begin{frame}[fragile]{}
\textbf{Linear Advection Equation }
\end{frame}

\begin{frame}[fragile]{Linear Advection Equation}
Consider the linear advection equation,
\begin{equation}
\frac{\partial u}{\partial t}  + v \frac{\partial u}{\partial x} = 0\,\,.
\end{equation}
This is the simplest example of a hyperbolic equation and its exact
solution is simply given by
\begin{equation}
u(t,x) = u(t=0,x-vt)\,\,,
\label{eq:pde_advect1}
\end{equation}
hence, the advection equation does (as one might expect) just translate
the given data along the x-axis with constant advection velocity $v$.
\end{frame}

\subsection{Solution Methods for the Linear Advection Equation}
\begin{frame}[fragile]{FTCS Discretization}
First-order in Time, Centered in Space (FTCS) Discretization gives
\begin{equation}
  u^{(n+1)}_j = u^{(n)}_j - \frac{v \Delta t}{2\Delta x} \left(u^{(n)}_{j+1} - u^{(n)}_{j-1}\right)\,\,.
\end{equation}
\end{frame}

\begin{frame}[fragile]{FTCS Discretization}
Introducing $u(x,t^{n}) = e^{ikx}$ into the discretized  equation,
\begin{equation}
\begin{aligned}
u^{(n+1)}_j &= e^{ik\Delta x j} - \frac{v \Delta t}{2\Delta x} \left(e^{ik\Delta x (j+1)} - e^{ik\Delta x (j-1)}\right)\,\,,\\
&= \left(1 - \frac{v\Delta t}{2 \Delta x} \left( e^{ik\Delta x} - e^{-ik\Delta x} \right) \right) e^{ik\Delta x j}\,\,,\\
&= \underbrace{\left( 1- \frac{v\Delta t}{\Delta x} i \sin(k\Delta x) \right)}_{\parbox{3cm}{\centering $=\xi$}} e^{ik\Delta x j}\,\,,
\end{aligned}
\end{equation}
and 
\begin{equation}
|\xi| = \sqrt{\xi \xi^*} = \sqrt{1 + \left(\frac{v\Delta t}{\Delta x} \sin(k\Delta x)\right)^2} > 1\,\,.
\end{equation}
Hence, the FTCS method is \emph{unconditionally unstable} for the
advection equation!
\end{frame}

\begin{frame}[fragile]{Upwind Method}
We consider now  a first-order in space, first-order in
time method to test if it could be stable. \\
The spatial
derivative, to first order, has two possiblities:
\begin{equation}
\begin{aligned}
\frac{\partial u}{\partial x} &\approx \frac{1}{\Delta x}(u_j - u_{j-1}) \hspace{1cm}\text{upwind finite difference,}\\
\frac{\partial u}{\partial x} &\approx \frac{1}{\Delta x}(u_{j+1} - u_{j})
\hspace{1cm}\text{downwind finite difference.}
\end{aligned}
\end{equation}
These are both so-called one-sided approximations, because they use
data only from one side or the other of point $x_j$. 
\end{frame}

\begin{frame}[fragile]{Upwind Method}
Coupling one
of these equations with forward Euler time integration yields
\begin{equation}
\begin{aligned}
u_j^{(n+1)} &= u_j^{(n)} - \frac{v\Delta t}{\Delta x} \left(u_j^{(n)} - u_{j-1}^{(n)}\right) \hspace{1cm}\text{upwind,}\\
u_j^{(n+1)} &= u_j^{(n)} - \frac{v\Delta t}{\Delta x} \left(u_{j+1}^{(n)} - u_{j}^{(n)}\right) \hspace{1cm}\text{downwind}\,\,.
\end{aligned}
\end{equation}
\end{frame}

\begin{frame}[fragile]{Upwind Method}
Stability analysis shows that the upwind method is stable for
\begin{equation}
0 \le \frac{v \Delta t}{\Delta x} \le 1\,\,,
\end{equation}
while the downwind method is stable for
\begin{equation}
-1 \le \frac{v \Delta t}{\Delta x} \le 0\,\,,
\end{equation}
which just confirms that for $v > 0$ one should use the upwind
method and for $v < 0$ the downwind method.
\end{frame}

\begin{frame}[fragile]{Upwind Method}
The condition
\begin{equation}
\alpha = \left|\frac{v\Delta t}{\Delta x}\right| \le 1\,\,
\label{eq:pde_courant1}
\end{equation}
 is a mathematical realization of the
physical causality principle: the propagation of information
(via advection) in one time step $\Delta t$ must not jump
ahead more than one grid interval of size $\Delta x$.

The demand that $\alpha \le 1$ is usually referred to as the 
\emph{Courant-Friedrics-Lewy (CFL) condition}. The
CFL condition is used to determine the  allowable time
step for a spatial grid size $\Delta x$ for a certain accuracy and
stability, 
\begin{equation}
\Delta t = c_\mathrm{CFL} \frac{\Delta x}{|v|}\,\,,
\end{equation}
where $c_\mathrm{CFL} \le 1$ is the CFL factor.
\end{frame}

\begin{frame}[fragile]{Lax-Friedrich Method}
The Lax-Friedrichs method, like FTCS, is first order in time, but
second order in space. It is given by
\begin{equation}
u_{j}^{(n+1)} = \frac{1}{2}\left(u^{(n)}_{j+1} + u^{(n)}_{j-1}\right) - 
\frac{v \Delta t}{2\Delta x} \left(u^{(n)}_{j+1} - u^{(n)}_{j-1}\right)\,\,,
\end{equation}
and, compared to FTCS, has been made stable for $\alpha \le 1$
(Eq.~\ref{eq:pde_courant1}) by using the average of the old result at
points $j+1$ and $j-1$ to compute the update at point $j$. This is
equivalent to adding a dissipative term to the equations (which damps
the instability of the FTCS method) and leads to rather poor accuracy.
\end{frame}

\begin{frame}[fragile]{Leapfrog Method}
A fully second-order method is the \emph{Leapfrog} Method (also called the
midpoint method) given by
\begin{equation}
u_j^{(n+1)} = u_j^{(n-1)} - \frac{v \Delta t}{\Delta
x} \left(u^{(n)}_{j+1} - u^{(n)}_{j-1}\right)\,\,.
\end{equation}
One can show that it is stable for $\alpha < 1$
(Eq.~\ref{eq:pde_courant1}).  A special feature of this method is that
it is \emph{non-dissipative}. This means that any initial condition
simply translates unchanged. All modes (in a decomposition picture) travel
unchanged, but they do not generally travel at the correct speed, which
can lead to high-frequency oscillations that cannot damp (since there
is no numerical viscosity).
\end{frame}

\begin{frame}[fragile]{Lax-Wendroff Method}
The \emph{Lax-Wendroff} method is an extension of the Lax-Friedrich method
to second order in space and time. It is given by
\begin{equation}
u_j^{(n+1)} = u_j^n - \frac{v\Delta t}{2\Delta x} \left(u_{j+1}^{(n)}
- u_{j-1}^{(n)} \right) + \frac{v^2(\Delta t)^2}{2(\Delta x)^2} \left(
u_{j-1}^{(n)} - 2 u_j^{(n)} + u_{j+1}^{(n)}\right)\,\,.
\end{equation}
It has considerably better accuracy than the Lax-Friedrich method and is
stable for $\alpha \le 1$ (Eq.~\ref{eq:pde_courant1}).
\end{frame}


\begin{frame}[fragile]{Next Class}
Poisson Equation
\end{frame}

\end{document}
