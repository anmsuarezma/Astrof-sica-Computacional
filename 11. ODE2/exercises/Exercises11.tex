\documentclass[11pt]{article}
\pagestyle{plain}
\usepackage{latexsym,exscale,amsfonts,amsmath,amssymb,array}
\usepackage{color}
\usepackage[colorlinks]{hyperref}
\setlength{\topmargin}{-2.3cm}
\setlength{\textheight}{23.8cm}
\setlength{\oddsidemargin}{-0.5cm}
\setlength{\textwidth}{17cm}
\setlength{\parindent}{0cm}
\setlength{\parskip}{.4cm}
\newcommand{\totaldiffx}{\frac{d}{dx}}
\newcommand{\pardiffx}{\frac{\partial}{\partial x}}
\newcommand{\luft}{\:\!}

\usepackage{graphicx}
\usepackage[latin1]{inputenc}
\usepackage{mathpazo}
\usepackage[T1]{fontenc}
\usepackage[comma,numbers,sort&compress]{natbib}


\begin{document}
\begin{center}
\large \bf Computational Astrophysics \rm \\
2019\\
{\small Exercises 11. ODE. Boundary values.}
\end{center}

 {\bf A simple Boundary-Value Problem}

Consider the Boundary-Value Problem (BVP)

\begin{equation}
\frac{d^2 y}{dx^2} = 12x - 4\,,\hspace*{0.3cm}y(0) = 0\,,\,\hspace{0.3cm}y(1) = 0.1\,\,.
\end{equation}

{\bf (1)} Solve the BVP ODE with the shooting method.

First, we need to rewrite the second-order ODE as two first-order ODEs:
\begin{align}
\frac{dy}{dx} &= u(x)\,\,,\nonumber\\
\frac{du}{dx} &= 12x - 4\,\,.
\end{align}
This is a really simple BVP and therefore it will require only very few
shooting iterations for reasonable initial guesses. Hence use the initial values $u(0) = z_0 = -1100000.0$
and $u(0) = z_1 = -10000000.0$ as initial guesses so that you get to do at least two
iterations of shooting. 

In the code directory you will find  a skeleton that you may use.  A
forward Euler integrator is already implemented. Solve the problem
by implementing an RK2 (or higher) integrator. Demonstrate convergence. Make a
plot of your solution for $y(x)$ and compare it with the true solution
is $y(x) = 2x^3 - 2x^2 + 0.1x$.


\noindent {\bf (2)} Solve the BVP
ODE with the finite-difference  method.
  

Happy Coding :) !

\end{document}
