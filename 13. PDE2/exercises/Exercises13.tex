\documentclass[11pt]{article}
\pagestyle{plain}
\usepackage{latexsym,exscale,amsfonts,amsmath,amssymb,array}
\usepackage{color}
\usepackage[colorlinks]{hyperref}
\setlength{\topmargin}{-2.3cm}
\setlength{\textheight}{23.8cm}
\setlength{\oddsidemargin}{-0.5cm}
\setlength{\textwidth}{17cm}
\setlength{\parindent}{0cm}
\setlength{\parskip}{.4cm}
\newcommand{\totaldiffx}{\frac{d}{dx}}
\newcommand{\pardiffx}{\frac{\partial}{\partial x}}
\newcommand{\luft}{\:\!}

\usepackage{graphicx}
\usepackage[latin1]{inputenc}
\usepackage{mathpazo}
\usepackage[T1]{fontenc}
\usepackage[comma,numbers,sort&compress]{natbib}


\begin{document}
\begin{center}
\large \bf Computational Astrophysics \rm \\
2019\\
{\small Exercises 13. Elliptic PDEs. Poisson Equation}
\end{center}

\textbf{Poisson Equation} \\
In these exercises you will solve the Poisson equation,
\begin{equation}
\nabla^2 \Phi = 4\pi G \rho,
\end{equation}
assuming spherical symmetry. 

\textbf{Homogeneous Sphere}\\
Consider the Poisson equation as a
  second-order ODE  and reduce it
  to first order (as described in class). 
\begin{enumerate}
\item Integrate it with forward Euler 
  using an appropriate initial condition.  Remember, that
  the gravitational potential is determined only up to a constant, so
  you will need to adjust your result to match the boundary condition at $R=10^9 \text{ cm}$,
  \begin{equation}
    \Phi(R) = -\frac{G M(R)}{R}\,\,.
  \end{equation}

  Since you will have to compute the mass on your grid,
  assume that you are solving for the gravitational
  potential of a homogeneous sphere ($\rho = const$), which is given by
  \begin{equation}
    \Phi(r) = \frac{2}{3}\pi G \rho  (r^2 - 3 R^2)\,\,.
  \end{equation}
  Show that your code converges to the exact value of $\Phi$ at
  $R_\text{outer}$ (the relative error must converge to zero at the
  right rate). Doing this is a bit tricky and you must be careful to
  adjust (via an additive constant) your solution at $r=0$ (where your
  $\Phi$ will be zero).

\item Now you'll try the more sophisticated matrix method. 
Write a code to solve Poisson equation via the linear system
\begin{equation}
J\mathbf{\Phi} = \mathbf{b}\,\,,
\end{equation}
as described in class. Check that your implementation works for the 
particular case $\rho = const$. 
\end{enumerate}
  
  
 \textbf{Stellar Model} \\
A supernova is a process in which a dying star liberates a huge quantity of energy and mass.\\
 In this section we will solve Poisson equation to obtain the gravitational potential inside the mass distribution of a dying star. \\
In the data directory you will find a file with data called {\tt presupernovaStar.dat}. The columns in this file are:
\begin{itemize}
\item First column: grid point index
\item Second column: enclosed mass of the star ($\text{gr}$)
\item Third column: radius of the distribution ($\text{cm}$)
\item Fourth column: temperature ($\text{K}$)
\item Fifth column: mass density ($\text{gr/cm}^3$)
\end{itemize}

\begin{enumerate}
\item Read the data and make a log-log plot of
  $\rho(r)$ for radii $r < 10^9 \,\mathrm{cm}$. 

\item Set up an equidistant grid with an outer radius of
  $10^9\,\mathrm{cm}$ and interpolate the density onto your new
  equidistant grid. You may choose what interpolation method to
  use.

\item Use the code that you write in the first exercise to obtain and plot 
the gravitational potential as a function of radius, solving the ODEs system.

\item Use the code to solve the Poisson equation using the matrix method.
Remember that you must stagger your
grid (shift it by $0.5 dx$ to make sure you have no point at the origin).
Obtain again $\Phi$ as a function of radius. The gravitational potential obtained
with this method must match (within errors) the potential obtained with
the ODE method.  Plot both in the same figure. 

\end{enumerate}

Happy Coding :) !


\end{document}
