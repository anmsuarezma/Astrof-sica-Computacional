\documentclass[11pt]{article}
\pagestyle{plain}
\usepackage{latexsym,exscale,amsfonts,amsmath,amssymb,array}
\usepackage{color}
\usepackage[colorlinks]{hyperref}
\setlength{\topmargin}{-2.3cm}
\setlength{\textheight}{23.8cm}
\setlength{\oddsidemargin}{-0.5cm}
\setlength{\textwidth}{17cm}
\setlength{\parindent}{0cm}
\setlength{\parskip}{.4cm}
\newcommand{\totaldiffx}{\frac{d}{dx}}
\newcommand{\pardiffx}{\frac{\partial}{\partial x}}
\newcommand{\luft}{\:\!}

\usepackage{graphicx}
\usepackage[latin1]{inputenc}
\usepackage{mathpazo}
\usepackage[T1]{fontenc}
\usepackage[comma,numbers,sort&compress]{natbib}


\begin{document}
\begin{center}
\large \bf Computational Astrophysics \rm \\
2019\\
{\small Exercises 10. Ordinary Differential Equations (ODEs)}
\end{center}

 {\bf ODE Integration:  Stellar Structure} \\
 
Consider the system of two ODEs describing a simple model of stellar structure
\begin{align}
\frac{dP}{dr} &= - \frac{GM(r)}{r^2}\, \rho\,\,, &\hspace{1cm}
\text{Hydrostatic Equilbrium}\\
\frac{dM}{dr} &= 4\pi \rho r^2\,\,. &\hspace{1cm} \text{Mass Conservation}
\end{align}

To complete the set of equations also requires an Equation Of State (EOS), i.e. a
relation between density and pressure. Since we will not solve an
internal energy equation, we will use a simplified EOS that is
barotropic: $P=P(\rho)$.  A possible choice is the polytropic EOS,
\begin{equation}
P = K \rho^\Gamma\,\,,
\end{equation}
where $K$ is the polytropic constant and $\Gamma$ is the adiabatic
index, i.e. the ratio of the specific heats. \\
A good kind of star to model
with a barotropic EOS is a white dwarf, which is supported by the
pressure of degenerate electrons.  In the case of a fully
relativistically degenerate white dwarf, we have the values
\begin{equation}
\begin{aligned}
K &= 1.244 \times 10^{15} (0.5)^\Gamma \text{dyne cm}^{-2}\,\, (g^{-1} \mathrm{cm^3})^\Gamma\\
\Gamma &= 4/3\,\,.
\end{aligned}
\end{equation}

In this exercise, you will solve the simplified stellar
structure equations by integrating them from the origin to the stellar
surface using Forward Euler, RK2 and, optionally also RK3, and
RK4. You will demonstrate convergence. 


%% {\bf Hints on the solution of the above equations}
%% \begin{itemize}
%% \item You will first need to setup a grid from $r=0$ to 
%%  some $r = r_\mathrm{max}$. Make it uniform.
%% \item This is an IVP and it is natural to specify boundary conditions
%%   at the origin. Be careful in doing this, since $r^{-2}$ is singular
%%   at $r=0$. You need a special case for $r = 0$ in the RHS routine.
%% \item You will need to compute the density from the pressure each time
%%   the RHS routine is called. You do this by inverting the EOS.
%% \item Mark the point at which the pressure drops below some prespecified
%%   fraction of the central pressure as the surface of the star.
%% \end{itemize}


\begin{enumerate}
\item In the code directory  you will finde a code skeleton for you to work
  from. Download it and study its structure.

\item Fill in the missing code segments (
  marked with  {\tt [ FILL IN CODE ]} ). You will have to implement
  the grid setup, the right-hand-side (RHS) for both equations,
  the Euler integrator, and the EOS inversion to obtain a new density
  from the updated pressure.

  Use a central density of $\rho_c = 10^{10}\,
  \mathrm{g}\,\mathrm{cm}^{-3}$ and a cut-off pressure corresponding
  to  $10^{-10} P_c$.  Test the code using 
  $\sim$1000 points and an outer radius of 2000 km.  You should get
  something close to $1.45\,M_\odot$ and a radius of
  $\sim$ $1500\,\mathrm{km}$.

\item Add an RK2 integrator and show convergence on
  the mass of the star using multiple resolutions. \\

  Implement also an RK3 or RK4 integrator.

\item

  Make a plot of $\rho(r)$ and $P(r)$ and $M(r)$ of a well-resolved
  RK2 (or RK3 or RK4) run. By intelligently rescaling them and using two
  $y$-axes, you can fit all of them on one plot. You may want to use a
  logarithmic scale for density and/or radius.
  
\end{enumerate}

Happy Coding :) !

\end{document}
